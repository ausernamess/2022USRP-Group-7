\documentclass[12pt,a4paper]{article}
\usepackage[margin=2cm]{geometry}
\usepackage{indentfirst}
\usepackage{mathrsfs}
\usepackage{amsthm}
\usepackage{CJKutf8}
\usepackage{amssymb}
\usepackage{amsmath}
\usepackage{import}
\usepackage{xifthen}
\usepackage{pdfpages}
\usepackage{transparent}
\usepackage{overpic}
\usepackage{setspace}
\usepackage{mathtools}
\usepackage{mathabx}
\usepackage{cases}
\usepackage{dsfont}
\usepackage{boondox-calo}
\usepackage{mdsymbol}
\usepackage{indentfirst}
\usepackage[utf8]{inputenc}
\usepackage[T1]{fontenc}
\usepackage{xcolor}

\pagestyle{empty}
\usepackage{titlesec}
\newcommand{\bbar}[1]{\overline{#1}}
\newcommand{\ul}[1]{\underline{#1}}
\newcommand{\SOL}{\fbox{ \tt s\parbox[b][2pt][c]{6pt}{o}\hspace*{-7pt} L:}}
\newcommand{\incfig}[1]{%
\import{./}{#1.pdf_tex}
}
\theoremstyle{plane}
\titleformat{\section}[block]{\color{black}\Large\bfseries\filcenter}{}{1em}{}
\newtheorem*{thm}{Theorem}
\newtheorem*{cor}{Corollary}
\newtheorem*{claim}{Claim}
\newtheorem*{prop}{Proposition}
\newtheorem*{lemma}{Lemma}
\newtheorem*{application}{Application}
\theoremstyle{definition}
\newtheorem{case}{Case}
\newtheorem*{defn}{Definition}
\newtheorem*{remark}{Remark}

\title{Trees and Double Phase Transition}
\author{Le-Rong Hsu}
\date{Augest 2022}
\begin{document}
\section*{Double Phase Transition}

In this section, we introduce the interesting property, double phase transition of $\mathbb{T}_d\times\mathbb{Z}$, which is different from integers lattices $\mathbb{Z}^d$, triangle lattices and regular trees. The number of infinite clusters changes on $[0,1]$; there's no infinite cluster on $(0,p_c)$, infinitely many on $(p_c,p_u)$ and one on $(p_u,1)$, where $p_u$ is defined as \[p_u=\{p\in [0,1] \mid \text{there is a.s. a unique infinite cluster}\}.\]
The property is given by the theorem:
\begin{thm}
For $d\geq 6$, \[0<p_c(\mathbb{T}_d\times\mathbb{Z})<p_u(\mathbb{T}_d\times\mathbb{Z})<1.\]
\end{thm}
For the lower bound of $p_c>0$, we can use the following lemma:
\begin{lemma}
For an infinite connected graph $G$, $p_c\geq\frac{1}{\mu}$, where $\mu$ is the connective constant of $G$.
\end{lemma}
The main idea is through bounding the $\theta(p)$ by the number of self-avoiding walks of length $n$ with probability $p^n$.\\To make sure $p_c<1$, 
we can use 
\begin{thm}
If $G$ is Cayley graph of a group with exponential growth, then $p_c<1$.
\end{thm}
where we bound the $p_c$ of $G$ by the $p_c$ of its subgraph, lexicographically minimal spanning tree. As for the inequality $p_u<1$, we apply the theorem showed by Babson and Benjamini
\begin{thm}
If $G$ is the Cayley graph of a nonamenable finitely presented group with one end, then $p_u<1$.
\end{thm}
where we consider special graphs and a combinatorial fact to obtain the desired result. The most difficult part is to show the inequality $p_c<p_u$. The essential ingredients are the following:
\begin{thm}
If $G$ is a $d-$regular connected multigraph, then \[\text{cogr}(G)>\sqrt{d-1} \text{ iff } \rho(G)>\frac{2\sqrt{d-1}}{d}\] in which case \[d\rho(G)=\frac{d-1}{\text{cogr}(G)}+\text{cogr}(G).\]
\end{thm}
\begin{cor}
For all $b\geq 1$, we have \[\rho(\mathbb{T}_d\times\mathbb{Z})=\frac{2\sqrt{b}+2}{b+3}.\]
\end{cor}
Of course, the proofs are nontrivial but they give the interesting property that the number of infinite clusters varies on the interval in $p$. \\
\bigskip\\
reference: R. Lyons, Y. Peres, \textit{Probability on Trees and Networks.} (2016)

\end{document}
