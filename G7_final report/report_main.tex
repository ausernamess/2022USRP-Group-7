\documentclass[12pt]{article}
\input{header.tex}
\input{./ListStyle.tex}
\author{Chia-Cheng, Hao \quad TBD \quad TBD \quad TBD \quad TBD \quad TBD}
\newcommand{\Instuition}{2022 NCTS USRP Group 7}
%\newcommand{\SeasonYear}{}
\newcommand{\Course}{Planar Statistical Physics and Bernoulli Percolation}
\newcommand{\Title}{Final Report}
%\newcommand{\Group}{7}
%\newcommand{\Uniquname}{}
%\newcommand{\UMID}{5950 8197}
\newcommand{\Instructor}{Prof. Jhih-Huang Li(NTU), Prof. Wai-Kit Lam(NTU)}
%\newcommand{\DueDate}{\DTMdisplaydate{2021}{9}{27}{-1}}
\usepackage{graphicx}


\begin{document}

\clearpage\maketitle
\thispagestyle{empty}

\newpage
\setcounter{page}{1}
%----------------------------------------------------------------------------------------------------------------------------------------------------
\section*{Abstract}
We study the percolation phenomenon in planar statistical physics using probability theory tools. We especially focus on \textit{Bernoulli percolation model}. 
In the first three weeks, we received classes covering four main topics. Those classes introduce methods to discuss the connecting property and phase transition behavior on some regular lattice such as $\mathbb{Z}^2,\mathbb{T}_d$, triangular lattice or hexagon lattice. 
After the classes were over, we went on individual research to explore topics such as exponential decay near critical probability and scaling invariant property of the crossing events.
\section*{Course Progress}
\subsection*{(Planer) Statistical Mechanics}
\subsection*{Planar Statistical Physics and Bernoulli Percolation}
\subsection*{Useful Identities \& Applications}
\subsection*{Exponential Decay}
\subsection*{Russo-Seymour-Welsh Theorem}
\section*{Individual Research}
\subsection*{Hao: Exponential Decay}
\section*{Future Work}

There are several directions in the future. For example, on the scaling invariant property, RSW theory gave us a way to obtain the uniform probability bounds of crossing events, but we can also ask the question about the scaling limit of a crossing event, not only the uniform bounds. We'll try to apply the discrete analytic ideas developed by Simrnov to extend the scaling problem on different types of lattice.

\section*{Reference}

\begin{enumerate}
\item S. Smirnov, \textit{Critical percolation in the plane: Conformal invariance, Cardy's formula, scaling limits, C. R. Acad. Sci. Paris} (2001).
\item Hugo Duminil-Copin, \textit{Introduction to Bernoulli percolation}, (2018).
\item Geoffrey R. Grimmett, Ioan Manolescu, \textit{Universality for bond percolation in two dimensions, Ann. Probab.} (2013).
\item R. Lyons, Y. Peres, \textit{Probability on Trees and Networks.} (2016)
\end{enumerate}





\end{document}