\documentclass[12pt]{article}
\input{header.tex}
\input{./ListStyle.tex}
\author{Chia-Cheng, Hao \quad Le-Rong, Hsu \quad TBD \quad TBD \quad TBD \quad TBD}
\newcommand{\Instuition}{2022 NCTS USRP Group 7}
%\newcommand{\SeasonYear}{}
\newcommand{\Course}{Planar Statistical Physics and Bernoulli Percolation}
\newcommand{\Title}{Final Report}
%\newcommand{\Group}{7}
%\newcommand{\Uniquname}{}
%\newcommand{\UMID}{5950 8197}
\newcommand{\Instructor}{Prof. Jhih-Huang Li(NTU), Prof. Wai-Kit Lam(NTU)}
%\newcommand{\DueDate}{\DTMdisplaydate{2021}{9}{27}{-1}}
\usepackage{graphicx}


\begin{document}

\clearpage\maketitle
\thispagestyle{empty}

\newpage
\setcounter{page}{1}
%----------------------------------------------------------------------------------------------------------------------------------------------------
\section{Abstract}
We study the percolation phenomenon in planar statistical physics using probability theory tools. We especially focus on \textit{Bernoulli percolation model}. 
In the first three weeks, we received classes covering four main topics. Those classes introduce methods to discuss the connecting property and phase transition behavior on some regular lattice such as $\mathbb{Z}^2,\mathbb{T}_d$, triangular lattice or hexagon lattice. 
After the classes were over, we went on individual research to explore topics such as exponential decay near critical probability and scaling invariant property of the crossing events.
\section{Course Progress}
\setcounter{subsection}{-1}
\subsection{(Planer) Statistical Mechanics}

%----------
\subsection{Planar Statistical Physics and Bernoulli Percolation}
Given an infinite undirected connected graph $G$. We say an edge $e$ is open(resp., closed) if and only if the edge $e$ is assigned the value $1$(resp., $0$). The name "Bernoulli" is given because we put i.i.d. Bernoulli random variables on each edge and each edge opens with probability $p\in[0,1]$. A configuration $\omega$ is an event that each edge $e\in \text{E}(G)$ is either open or closed. Two vertices $a$ and $b$ are connected if there exists a path from $a$ to $b$. A cluster is a set of vertices that every pair of vertices is connected. If the clusters is infinite, then we say
that it is an infinite cluster. We define a function $\theta(p)=\mathbb{P}_p(\exists\text{ an infinite cluster containing }0)$, that is, the probability of the event in the parentheses under the Bernoulli percolation with probability $p$. The key ingredient of the Bernoulli percolation is the critical probability $p_c\coloneqq\\sup\{in [0,1]\mid \theta(p)=0\}$, under which the behavior of percolation changes significantly. The focus of our course is the existence and uniqueness of the infinite cluster and the behavior of the percolation when $p>p_c$ and $p<p_c$. 

%----------
\subsection{Useful Identities \& Applications}

%----------
\subsection{Exponential Decay}\label{Course_Progress:Exponentia_Decay}
One of the reason we introduce exponential decay is to show that $p_c \leq \frac{1}{2}$ on $\mathbb{Z}^d$(shown by Harry Kesten in the 80s). 
For $\mathbb{Z}^d$ with $d\geq 1$, depending on whether $p$ is larger than $p_c$, the probability of the origin connected to $\partial \Lambda_n$  ($0 \longleftarrow \partial \Lambda_n$) will either decay exponentially as $n$ grows or bounded below by a constant depending on p. 
Once knowing the above fact, we can show that $\mathbb{P}_p (\mathcal{H}_n)$, the probability of horizontal crossing on $R_n$($[0,n]\times [0,n+1]$) box, also decrease exponentially as $n$ grows under $p < p_c$. 
In the discussion of $\mathbb{Z}^2$, if $p_c < \frac{1}{2}$, we will get a contradiction to $\mathbb{P}_{\frac{1}{2}}(\mathcal{H}_n) = 1/2$ for any $n \in \mathbb{N}$. Therefore, combining the previous discussion we know $p_c = \frac{1}{2}$ on $\mathbb{Z}^2$.

We can observe exponential decay on other event. For example, the size of $C$, where $C$ is the connected component containing the origin, and the probability of the origin connected to a point on one axis with distance $n$ ($0 \longleftarrow ne_1$). In the latter event, we applied the extension of Fekete subadditive lemma to get the parameter for the decay, \textit{correlation length} $\xi(p)$. 
Using the same method, the rate of decay $\varphi(p)$ for $0 \longleftarrow \partial \Lambda_n$ is also acquired, and we know $\varphi(p) = \xi(p)^{-1}$.

%----------
\subsection{Russo-Seymour-Welsh Theorem}
In this section, we discuss the scale invariant property of some connecting events when $p=p_c$ on $\mathbb{Z}^2$ lattice. In particular, we presents the invariant behaviour of the \textit{horizontal crossing event in a rectangle of size $[0,\rho n]\times[0,n]$}, which is denoted by $\mathcal{H}(\rho n, n)$.\\ \\
\textbf{Theorem (Russo-Seymour-Welsh).}
\textit{Let $\rho>0$. There exists $c=c(\rho)>0$ such that for all $n\geq 1$, we have}
\begin{equation*}
c\leq\mathbb{P}_{\frac{1}{2}}\big[\mathcal{H}(\rho n, n)\big]\leq 1-c,
\end{equation*}
\textit{where $\mathcal{H}(\rho n, n)$ denotes the horizontal crossing event in a rectangle with size $[0,\rho n]\times [0,n]$.}\\ \\
They proved this theorem by proving a special case:\\
\textbf{Theorem.} \textit{For all }$n\geq 1$,
\begin{equation*}
\mathbb{P}_{\frac{1}{2}}\big[\mathcal{H}(3n,2n)\big]\geq\frac{1}{128}.
\end{equation*}\\
Once one get this result, i.e. when one find $c(\rho)$ for some $\rho>1$ (e.g. $c(3/2)$), then one can get $c(\rho')$ for arbitrary $\rho'>1$ by construct the crossing events $\mathcal{H}(\rho n,n)$ that assures $\mathcal{H}(\rho'n, n)$ to occur, and hence we can prove the first theorem. For example, to get a lower bound for $\mathbb{P}_{p_c}[\mathcal{H}(4n,n)]$, we can place five $(2n,n)$ boxes as follows: Let $R_1=[0,2n]\times[0,n]$, $R_2=[n,2n]\times[-n,n]$, $R_3=[n,3n]\times[-n,0]$, $R_4=[2n,3n]\times[-n,n]$ and $R_5=[2n,4n]\times[0,n]$. Then we have
\begin{equation*}
\mathbb{P}_{p_c}[\mathcal{H}(R_1)\cap\mathcal{V}(R_2)\cap\mathcal{H}(R_3)\cap\mathcal{V}(R_4)\cap\mathcal{H}(R_5)]\leq\mathbb{P}_{p_c}[\mathcal{H}([0,4n]\times[0,n])].
\end{equation*}
Now by Harris-FKG inequality and translation invariant property on $\mathbb{Z}^2$ lattice, we immediately have
\begin{equation*}
c(2)^5\leq \mathbb{P}_{p_c}[\mathcal{H}(4n,n)].
\end{equation*}

With Russo-Seymour-Welsh's theory, we're able to give a scale invariant property for more general crossing events. Consider a simply connected domain with a smooth boundary $\Omega$ with distinct boundary points $a,b,c,d$. For $\delta>0$, we define a finite graph $\Omega^\delta=\delta\mathbb{Z}^2\cap\Omega$. And let $a^\delta, b^\delta, c^\delta, d^\delta\in\Omega^\delta$ to be the closest points to $a,b,c,d\in\partial\Omega$. Also define $(a^\delta b^\delta),\,(c^\delta d^\delta)$ as the paths on $\partial\Omega^\delta$ from $a^\delta$ to $b^\delta$, from $c^\delta$ to $d^\delta$ counterclockwise.\\ \\
\textbf{Theorem.} \textit{There exists $c=c(\Omega,a,b,c,d)>0$ such that for any $\delta>0$,}
\begin{equation*}
\mathbb{P}_\frac{1}{2}\big[(a^\delta b^\delta)\xleftrightarrow{\text{$\Omega^\delta$}}(c^\delta d^\delta)\big]\geq c.
\end{equation*}
    
\begin{figure}[b]
\centering
\includegraphics[width=5.0cm]{./RSWpart/omega_2.png}
\includegraphics[width=5.0cm]{./RSWpart/omega_2_crossing.png}
\end{figure}
    
    

\section{Individual Research}
\subsection*{Hao: Exponential Decay}
In \ref{Course_Progress:Exponentia_Decay}, we were left with on solved exercise 


\section{Future Work}

There are several directions in the future. For example, on the scaling invariant property, RSW theory gave us a way to obtain the uniform probability bounds of crossing events, but we can also ask the question about the scaling limit of a crossing event, not only the uniform bounds. We'll try to apply the discrete analytic ideas developed by Simrnov to extend the scaling problem on different types of lattice.

\section*{Reference}

\begin{enumerate}
\item S. Smirnov, \textit{Critical percolation in the plane: Conformal invariance, Cardy's formula, scaling limits, C. R. Acad. Sci. Paris} (2001).
\item Hugo Duminil-Copin, \textit{Introduction to Bernoulli percolation}, (2018).
\item Geoffrey R. Grimmett, Ioan Manolescu, \textit{Universality for bond percolation in two dimensions, Ann. Probab.} (2013).
\item R. Lyons, Y. Peres, \textit{Probability on Trees and Networks.} (2016)
\end{enumerate}





\end{document}
