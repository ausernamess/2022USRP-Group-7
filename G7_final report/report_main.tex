\documentclass[12pt]{article}
\input{header.tex}
\input{./ListStyle.tex}
\author{ Shi-Hong, Su\\ Yan-Wei, Su\\Chia-Cheng, Hao\\ Le-Rong, Hsu\\ Wei-Chieh, Hung\\}
\newcommand{\Instuition}{2022 NCTS USRP Group 7}
%\newcommand{\SeasonYear}{}
\newcommand{\Course}{Planar Statistical Physics and Bernoulli Percolation}
\newcommand{\Title}{Final Report}
%\newcommand{\Group}{7}
%\newcommand{\Uniquname}{}
%\newcommand{\UMID}{5950 8197}
\newcommand{\Instructor}{Prof. Jhih-Huang Li(NTU), Prof. Wai-Kit Lam(NTU)}
%\newcommand{\DueDate}{\DTMdisplaydate{2021}{9}{27}{-1}}
\newcommand{\sixedge}{\mbox{\parbox[t][-2pt][c]{10pt}{\scriptsize$\blacktriangledown\hspace{-3.6pt}\blacktriangle\hspace{-4pt}\blacktriangledown$}\hspace{-10.2pt}\parbox[b][10.7pt][c]{10pt}{\scriptsize$\blacktriangle\hspace{-3.6pt}\blacktriangledown\hspace{-3.6pt}\blacktriangle$}}}
\usepackage{graphicx}
\theoremstyle{plane}
\newtheorem*{thm}{Theorem}
\newtheorem*{cor}{Corollary}
\newtheorem*{claim}{Claim}
\newtheorem*{prop}{Proposition}
\newtheorem*{application}{Application}
\theoremstyle{definition}
\newtheorem{case}{Case}
\newtheorem*{defn}{Definition}
\newtheorem*{remark}{Remark}


\newcommand{\iso}{\mathrm{iso}}
\newcommand{\esc}{\mathrm{esc}}
\newcommand{\oversim}[1]{\parbox[t][-18pt][c]{10pt}{\scriptsize$\sim$}\hspace*{-12pt}{#1}}
\newcommand{\bbar}[1]{\overline{#1}}
\newcommand{\ul}[1]{\underline{#1}}
\newcommand{\SOL}{\fbox{ \tt s\parbox[b][2pt][c]{6pt}{o}\hspace*{-7pt} L:}}
\newcommand{\sixxedge}{\mbox{\parbox[t][-2pt][c]{10pt}{\scriptsize$\blacktriangledown\hspace{-3.6pt}\blacktriangle\hspace{-4pt}\blacktriangledown$}\hspace{-10.2pt}\parbox[b][10.7pt][c]{10pt}{\scriptsize$\blacktriangle\hspace{-3.6pt}\blacktriangledown\hspace{-3.6pt}\blacktriangle$}}}
\newcommand{\indecate}{\mbox{\begin{turn}{65.9}
$>$
\end{turn}\hspace{-10.5pt}\parbox[t][-5pt][c]{10pt}{$\bot$}\hspace{-4.35pt}\parbox[t][-3.4pt][c]{4pt}{$\shortmid$}}}
\newcommand{\inccfig}[1]{%
\import{./picture/}{#1.pdf_tex}
}

\begin{document}

\clearpage\maketitle
\thispagestyle{empty}

\newpage
\setcounter{page}{1}
%----------------------------------------------------------------------------------------------------------------------------------------------------
\section{Abstract}
We study the percolation phenomenon in planar statistical physics using probability theory tools. We especially focus on \textit{Bernoulli percolation model}. 
In the first three weeks, we received classes covering four main topics. Those classes introduce methods to discuss the connecting property and phase transition behavior on some regular lattice such as $\mathbb{Z}^2,\mathbb{T}_d$, triangular lattice or hexagon lattice. 
After the classes were over, we went on individual research to explore topics such as exponential decay near critical probability and scaling invariant property of the crossing events.

A more detailed version is on our \href{https://github.com/ausernamess/2022USRP-Group-7/blob/main/main.pdf}{Github} page. Reader who are interested in the topic can also check out Prof. Li's USRP webpage at \href{https://usrp2022.cadlag.space/}{here}
\section{Course Progress}

\setcounter{subsection}{-1}
\subsection{(Planer) Statistical Mechanics}
One motivation for the percolation models is statistical phenomenon. When trying to describe macroscopic behavior with a huge system of microscopic particles, it is nearly impossible to solve the system with so many variables.
In the late 19th century, Ludwig Boltzmann raised a probabilistic formation to these problems. In particular, Boltzmann distribution. 
Let $\Omega$ be our sample space, which is a collection of all possible configuration $\omega$ of the system. Then, the 'energy' of $\omega$ can be introduced as Hamiltonian $H(\omega)$, and $\beta$ is a parameter related to temperature.
The probability of $\omega$ happening in Boltzmann distribution can be denoted as 
$$
    \mathbb{P}_\beta(\omega) = \frac{1}{\mathbb{Z_\beta}}e^{-\beta H(\omega)}, 
$$ 
where $\mathbb{Z}_\beta = \sum_{\omega \in \Omega} e^{-\beta H(\omega)}$ is a normalization constant.
If $\Omega$ is a infinite set, the finiteness of $\mathbb{Z}_\beta$ is not guaranteed and depend on $\beta$.
The first model we consider is polymer/self-avoiding walk model on $\mathbb{Z}^2$. The model is attributed to Chemist Paul Flory in the 50s. A \textbf{polymer} is a self avoiding walk from the origin, and a \textbf{self-avoiding walk}(SAW) is a path that does not interest its trajectory.
Also, $H(\omega)= |\omega|$, i.e the energy of $\omega$ is determined by the length of the SAW. 
Then, 
$$
    \mathbb{Z_\beta} = \sum_{\omega \in \Omega} e^{-\beta H(\omega)} = \sum_{n \in \mathbb{N}} \lambda_n e^{-\beta n}, 
$$
where $\lambda_n$ is the number of $\omega$ with $|\omega| = n$.
To make the model well defined, we need to make $\mathbb{Z}_\beta < \infty$. Hence, we have to understand the asymptotic behavior of $\lambda_n$ and thus finding $\beta$ that make $\mathbb{Z}_\beta < \infty$.

The property we are looking for is the subadditivity of $\lambda_n$. This property gives the limit $\mu \coloneqq \lim_n \lambda_n^{\frac{1}{n}}$. The \textbf{connective constant} $\mu$ is hard to compute, but once acquired, we can define critical $\beta_c \coloneqq \ln{(\mu)}$.
When $\beta < \beta_c$, $\mathbb{Z}_\beta$ will be less than $\infty$.

%----------
\subsection{Planar Statistical Physics and Bernoulli Percolation}
Given an infinite undirected connected graph $G = (V,E)$. We say an edge $e \in E$ is open (resp., closed) if and only if the edge $e$ is assigned the value $1$(resp., $0$). The name "Bernoulli" is given because we put i.i.d. Bernoulli random variables on each edge and each edge opens with probability $p\in[0,1]$. A configuration $\omega$ is an event that each edge $e\in \text{E}(G)$ is either open or closed. Two vertices $a$ and $b$ are connected if there exists a path from $a$ to $b$. A cluster is a set of vertices that every pair of vertices is connected. If a cluster is infinite, then we say that it is an infinite cluster. We define a function $\theta(p)=\mathbb{P}_p(\exists\text{ an infinite cluster containing }0)$, that is, the probability of the event in the parentheses under the Bernoulli percolation with probability $p$. The key ingredient of the Bernoulli percolation is the critical probability \[p_c\coloneqq\\sup\{p\in [0,1]\mid \theta(p)=0\},\] under which the behavior of percolation changes significantly. The focus of our course is the existence and uniqueness of the infinite cluster and the behavior of the percolation when $p>p_c$ (supercritical) and $p<p_c$ (subcritical). 

%----------
\subsection{Useful Identities \& Applications}
We introduce some fundamental tools which are useful when we study the Bernoulli percolation on other graphs. One important idea is the increasing event. Intuitively, if $A$ is an increasing event, $\omega\in A$ and $\omega\leq\omega'$, then $\omega'$ is in $A$. One noteworthy thing is that we usually consider the increasing events that "depend on finitely many edges". This comes our first tool, Harris-FKG inequality.
\begin{thm}
If $A$ and $B$ are two increasing events, then $\mathbb{P}(A\cap B)\geq \mathbb{P}(A)\mathbb{P}(B)$.
\end{thm}
This is equivalent to 
\begin{thm}
If \(f\) and \(g\) are increasing functions, then $\mathbb{E}[fg]\geq\mathbb{E}[f]\mathbb{E}[g].$
\end{thm}
The Harris-FKG inequality can be applied to the proof that on $\mathbb{Z}^2$, $p_c(\mathbb{Z}^2)\geq 1/2$ with the square-root trick. Another important tool is the B.K. inequality.
\begin{thm}
If A and B are increasing events that depend on finitely many edges, we have \[\mathbb{P}(A\circ B)\leq\mathbb{P}(A)\mathbb{P}(B).\]
\end{thm}
The event $A\circ B$ denotes "$A$ and $B$ are realized disjointly". We could view $A\circ B$ as "the edges that $A$ and $B$ depend on are disjoint". The equality holds when the events $A$ and $B$ are disjoint. Intuitively, the B.K. inequality tells us that if an increasing event holds then it gets harder for another increasing event to hold disjointly. We can manage to use this inequality in order to bound the some probability. The last tool is Russo's identity. This is useful when we study of the behavior $\mathbb{P}_p$ when $p$ varies. Here, it's necessary to introduce the idea of 
pivotal edges. An edge $e$ is said to be pivotal of the configuration $\omega$ if the "switch" of $e$ determines whether $\omega$ is in the event $A$ or not. Write $N(A)=N(A,\omega)$ be the number of pivotal edges in $\omega$. 
\begin{prop}
For an increasing event $A$ depending on finitely many edges, we have \[\frac{d}{dp}\mathbb{P}_p (A)=\mathbb{E}_p[N(A)].\]
\end{prop}
The identity is essential for the study of behavior of supercritical regime. 
%----------
\subsection{Exponential Decay}\label{Course_Progress:Exponentia_Decay}
\begin{thm} For $d \geq 2$, there are two facts
\begin{enumerate}
	    \item For $p < p_c$, there exists $c = c(p) > 0$ s.t For all $n\geq 1$,
				$$
					\mathbb{P}_p[0 \longleftrightarrow \partial \Lambda_n] \leq e^{-cn}.	
				$$
		\item For $p > p_c$,
			$$
			\mathbb{P}[0\longleftrightarrow \infty ]\geq \frac{p-p_c}{p(1-p_c)}.
			$$
\end{enumerate}
\end{thm}
One of the reason we introduce exponential decay is to show that $p_c \leq \frac{1}{2}$ on $\mathbb{Z}^d$(shown by Harry Kesten in the 80s). 
For $\mathbb{Z}^d$ with $d\geq 1$, depending on whether $p$ is larger than $p_c$, the probability of the origin connected to $\partial \Lambda_n$  ($0 \longleftarrow \partial \Lambda_n$) will either decay exponentially as $n$ grows or bounded below by a constant depending on p. 
Once knowing the above fact, we can show that $\mathbb{P}_p (\mathcal{H}_n)$, the probability of horizontal crossing on $R_n$($[0,n]\times [0,n+1]$) box, also decrease exponentially as $n$ grows under $p < p_c$. 
In the discussion of $\mathbb{Z}^2$, if $p_c < \frac{1}{2}$, we will get a contradiction to $\mathbb{P}_{\frac{1}{2}}(\mathcal{H}_n) = 1/2$ for any $n \in \mathbb{N}$. Therefore, combining the previous discussion we know $p_c = \frac{1}{2}$ on $\mathbb{Z}^2$.

We can observe exponential decay on other event. For example, the size of $C$, where $C$ is the connected component containing the origin, and the probability of the origin connected to a point on one axis with distance $n$ ($0 \longleftarrow ne_1$). In the latter event, we applied the extension of Fekete subadditive lemma to get the parameter for the decay, \textit{correlation length} $\xi(p)$ as 
$$
\xi(p) \coloneqq \inf_{n} \left(-\frac{1}{n}\ln{\mathbb{P}_p[0 \leftrightarrow ne_1]}\right)^{-1}.
$$
Using the same method, the rate of decay $\varphi(p)$ for $0 \longleftarrow \partial \Lambda_n$ is also acquired, and we will be able to show that $\varphi(p) = \xi(p)^{-1}$.

%----------
\subsection{Russo-Seymour-Welsh Theorem}
In this section, we discuss the scale invariant property of some connecting events when $p=p_c$ on $\mathbb{Z}^2$ lattice. In particular, we presents the invariant behaviour of the \textit{horizontal crossing event in a rectangle of size $[0,\rho n]\times[0,n]$}, which is denoted by $\mathcal{H}(\rho n, n)$.\\[12pt]
\textbf{Theorem (Russo-Seymour-Welsh).}
\textit{Let $\rho>0$. There exists $c=c(\rho)>0$ such that for all $n\geq 1$, we have}
\[
c\leq\mathbb{P}_{\frac{1}{2}}\big[\mathcal{H}(\rho n, n)\big]\leq 1-c,
\]
\textit{where $\mathcal{H}(\rho n, n)$ denotes the horizontal crossing event in a rectangle with size $[0,\rho n]\times [0,n]$.}\\[12pt]
They proved this theorem by proving a special case:\\
\textbf{Theorem.} \textit{For all }$n\geq 1$,
\begin{equation*}
\mathbb{P}_{\frac{1}{2}}\big[\mathcal{H}(3n,2n)\big]\geq\frac{1}{128}.
\end{equation*}\\
Once one get this result, i.e. when one find $c(\rho)$ for some $\rho>1$ (e.g. $c(3/2)$), then one can get $c(\rho')$ for arbitrary $\rho'>1$ by construct the crossing events $\mathcal{H}(\rho n,n)$ that assures $\mathcal{H}(\rho'n, n)$ to occur, and hence we can prove the first theorem. For example, to get a lower bound for $\mathbb{P}_{p_c}[\mathcal{H}(4n,n)]$, we can place five $(2n,n)$ boxes as follows: Let $R_1=[0,2n]\times[0,n]$, $R_2=[n,2n]\times[-n,n]$, $R_3=[n,3n]\times[-n,0]$, $R_4=[2n,3n]\times[-n,n]$ and $R_5=[2n,4n]\times[0,n]$. Then we have
\begin{equation*}
\mathbb{P}_{p_c}[\mathcal{H}(R_1)\cap\mathcal{V}(R_2)\cap\mathcal{H}(R_3)\cap\mathcal{V}(R_4)\cap\mathcal{H}(R_5)]\leq\mathbb{P}_{p_c}[\mathcal{H}([0,4n]\times[0,n])].
\end{equation*}
Now by Harris-FKG inequality and translation invariant property on $\mathbb{Z}^2$ lattice, we immediately have
\begin{equation*}
c(2)^5\leq \mathbb{P}_{p_c}[\mathcal{H}(4n,n)].
\end{equation*}

With Russo-Seymour-Welsh's theory, we're able to give a scale invariant property for more general crossing events. Consider a simply connected domain with a smooth boundary $\Omega$ with distinct boundary points $a,b,c,d$. For $\delta>0$, we define a finite graph $\Omega^\delta=\delta\mathbb{Z}^2\cap\Omega$. And let $a^\delta, b^\delta, c^\delta, d^\delta\in\Omega^\delta$ to be the closest points to $a,b,c,d\in\partial\Omega$. Also define $(a^\delta b^\delta),\,(c^\delta d^\delta)$ as the paths on $\partial\Omega^\delta$ from $a^\delta$ to $b^\delta$, from $c^\delta$ to $d^\delta$ counterclockwise.\\[12pt]
\textbf{Theorem.} \textit{There exists $c=c(\Omega,a,b,c,d)>0$ such that for any $\delta>0$,}
\begin{equation*}
\mathbb{P}_\frac{1}{2}\big[(a^\delta b^\delta)\xleftrightarrow{\text{$\Omega^\delta$}}(c^\delta d^\delta)\big]\geq c.
\end{equation*}
    
\begin{figure}[h]
\centering
\includegraphics[width=5.0cm]{./picture/omega_2.png}
\includegraphics[width=5.0cm]{./picture/omega_2_crossing.png}
\end{figure}
\newpage
    

\section{Individual Research}
\subsection*{Hao: Exponential Decay}
I was intended to study the paper [3], and I found the idea of correlation length and the exponents are helpful in reading the paper. Fortunately, these two are part of \ref{Course_Progress:Exponentia_Decay}. 
The tree exercises I solved were the exponential decay on the size of the connected component $C$ under subcritical stage, correlation length, and Fekete subadditive lemma on $0 \longleftarrow \partial \Lambda_n$.
In solving these exercises, I have more understanding on the relation between different exponents and how can they help us on the study of near critical phenomenon.  
\bigskip
\subsection*{Hsu: Double Phase Transition}
In this section, we introduce the interesting property, double phase transition of $\mathbb{T}_d\times\mathbb{Z}$, which is different from integers lattices $\mathbb{Z}^d$, triangle lattices and regular trees. The number of infinite clusters changes on $[0,1]$; there's no infinite cluster on $(0,p_c)$, infinitely many on $(p_c,p_u)$ and one on $(p_u,1)$, where $p_u$ is defined as \[p_u=\{p\in [0,1] \mid \text{there is a.s. a unique infinite cluster}\}.\]
The property is given by the theorem:
\begin{thm}
For $d\geq 6$, \[0<p_c(\mathbb{T}_d\times\mathbb{Z})<p_u(\mathbb{T}_d\times\mathbb{Z})<1.\]
\end{thm}
For the lower bound of $p_c>0$, we can use the following lemma:
\begin{prop}
For an infinite connected graph $G$, $p_c\geq\frac{1}{\mu}$, where $\mu$ is the connective constant of $G$.
\end{prop}
The main idea is through bounding the $\theta(p)$ by the number of self-avoiding walks of length $n$ with probability $p^n$.\\To make sure $p_c<1$, 
we can use 
\begin{thm}
If $G$ is Cayley graph of a group with exponential growth, then $p_c<1$.
\end{thm}
where we bound the $p_c$ of $G$ by the $p_c$ of its subgraph, lexicographically minimal spanning tree. As for the inequality $p_u<1$, we apply the theorem showed by Babson and Benjamini
\begin{thm}
If $G$ is the Cayley graph of a nonamenable finitely presented group with one end, then $p_u<1$.
\end{thm}
where we consider special graphs and a combinatorial fact to obtain the desired result. The most difficult part is to show the inequality $p_c<p_u$. The essential ingredients are the following:
\begin{thm}
If $G$ is a $d-$regular connected multigraph, then \[\text{cogr}(G)>\sqrt{d-1} \text{ iff } \rho(G)>\frac{2\sqrt{d-1}}{d},\] in which case \[d\rho(G)=\frac{d-1}{\text{cogr}(G)}+\text{cogr}(G).\]
\end{thm}
\begin{cor}
For all $b\geq 1$, we have \[\rho(\mathbb{T}_d\times\mathbb{Z})=\frac{2\sqrt{b}+2}{b+3}.\]
\end{cor}
Of course, the proofs are nontrivial but they give the interesting property that the number of infinite clusters varies on the interval in $p$. 
\vspace{1cm}
\subsection*{Hung: Research of other periodic lattice}
\begin{enumerate}
    \item $\mathbb{T}_d:$
	\begin{figure}[htp]
	\centering
	\def\svgwidth{7cm}
	\inccfig{T3}
	\end{figure}\\
	We well find that either in bond percolation or in site percolation, $p_c(\mathbb{T}_d)=\frac{1}{d-1}$ and $ \mu(\mathbb{T}_d)=d-1.$
	\begin{proof}
	\begin{enumerate}
	    \item The idea of $\mu(\mathbb{T}_d)=d-1$ is easy to see because of the following recursive relation : 
	    \[
	    \lambda_n=\lambda_{n-1}\times(d-1).
	    \]
	    \item We define the connective graph $\mathbb{I}_d$ to be a connective component contain $O$ of a subgraph of $\mathbb{T}_d$ that cut an edge beside $O.$
    	\begin{figure}[htp]
    	\centering
    	\def\svgwidth{7cm}
    	\inccfig{I3}
    	\end{figure}\\
    	And then, we define a function $\beta:[0,1]\to [0,1]$ as $\beta(p)=\mathbb{P}_p\{O\leftrightarrow\infty\mbox{ in }\mathbb{I}_d\}.$\\
    	Last, using the fact that $p_c(\mathbb{T}_d)=p_c(\mathbb{I}_d)$ we can see that $p_c(\mathbb{T}_d)=\frac{1}{d-1}.$
	\end{enumerate}
	\end{proof}
	\item $\triangle:$ (triangle lattice)\\[5pt]
	We can see that in site percolation, $p_c(\triangle)\geq \frac{1}{2}.$ (Actually, $p_c(\triangle)=\frac{1}{2}.$)
	\begin{figure}[htp]
	\centering
	\def\svgwidth{7cm}
	\inccfig{nothing}
	\end{figure}\\
	To show this, we use similar way of showing $\theta_{\mathbb{Z}_2}(p)=0,$ using the idea of \textbf{uniqueness} and \textbf{dual graph}; It is needed to mention that in site percolation of triangular mesh we can define a ``dual" graph of a configuration $\omega$ by 
	\[
	\omega^*=\{v\in V(G)\mid \omega(v)=0\}
	\]
	Thus we can observe that either it has a horizontal crossing in $\omega$ or a vertical crossing in $\omega^*.$ Thus if $\theta(\frac{1}{2})>0,$ we can find that it contradicted to $\mathbb{P}_{\frac{1}{2}}(N=1)=1.$
\end{enumerate}
\vspace{1cm}
\subsection*{Su: Critical Probability for Bond Percolation on $\triangle$}
\begin{thm}
The critical probability for bond percolation on $\triangle$ is $2\sin (\frac{\pi}{18})$.
\end{thm}
Proof:
\begin{enumerate}
	\item Consider a single triangle $G$ and a defined graph $G'$ and try to find the relations between $G$ and $G'$.
	\begin{center}
		\includegraphics[width=9cm]{./picture/try_an_goal.png}
	\end{center}
	\item We obtain that the graph $G'$ conserves the connectivity on $G$ if the following holds: 
		\begin{itemize}
			\item $p_{G'} = 1 - p_{G}$.
			\item $p_{G}$ is the only solution between $[0,1]$ satisfying the formula $p^3-3p+1=0$.
		\end{itemize}
	\item By checking that both cases where $p_{c}^{\triangle} > p_{G}$ and $p_{c}^{\triangle} < p_{G}$ gives us contradiction, we eventually will have $p_{c}^{\triangle} = p_{G}$, where $p_{G}$, as the solution of the above formula, can be written in the form of $2\sin(\frac{\pi}{18})$.
\end{enumerate}

\newpage
\section{Future Work}

There are several directions in the future. For example, on the scaling invariant property, RSW theory gave us a way to obtain the uniform probability bounds of crossing events, but we can also ask the question about the scaling limit of a crossing event, not only the uniform bounds. We'll try to apply the discrete analytic ideas developed by Simrnov to extend the scaling problem on different types of lattice.
We can also back to the starting point, thinking the method of finding critical probability or connective constant in some periodic lattice (ex. $\triangle,\ \sixedge,$ etc.). 


\section*{Reference}

\begin{enumerate}
\item S. Smirnov, \textit{Critical percolation in the plane: Conformal invariance, Cardy's formula, scaling limits, C. R. Acad. Sci. Paris} (2001).
\item Hugo Duminil-Copin, \textit{Introduction to Bernoulli percolation}, (2018).
\item Geoffrey R. Grimmett, Ioan Manolescu, \textit{Universality for bond percolation in two dimensions, Ann. Probab.} (2013).
\item R. Lyons, Y. Peres, \textit{Probability on Trees and Networks.} (2016).
\end{enumerate}





\end{document}